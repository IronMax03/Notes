\documentclass[5pt]{article}
\usepackage{multicol,multirow}
\usepackage{graphicx} % Required for inserting images
\usepackage[margin=0.75cm]{geometry}
\usepackage{xcolor}
\usepackage{amsmath}
\usepackage{mathtools}

\usepackage{empheq}
\usepackage{amsfonts}

\definecolor{LightGray}{gray}{0.9}

\usepackage{minted}

\DeclarePairedDelimiter\abs{\lvert}{\rvert}%
\DeclarePairedDelimiter\norm{\lVert}{\rVert}%

\makeatletter
\let\oldabs\abs
\def\abs{\@ifstar{\oldabs}{\oldabs*}}


\begin{document}
\begin{center}
     \Large{\textbf{Calculus 1 For Engineers}}\\
     \small{Class: APPM 1350}\hfill\small{\textcopyright Maximilien Notz \the\year{}}
     \noindent\rule{20.2cm}{0.4pt}
\end{center}


\begin{multicols}{3}
\setcounter{secnumdepth}{0}

\section{Geometry\footnotesize{(Pre-Calculus)}}
\begin{tabular}{ll}
Circle Cir. & $2\pi r$\\
circle area & $A=\pi r^2$\\
Sphere area & $A=4\pi r^2$\\
Sphere Vol. & $V =\frac{4}{3}\pi r^3$\\[1pt]
Pyramid Vol. & $V =\frac{1}{3}a_{base} h$\\

\end{tabular}

\section{Trigonometric\footnotesize{(Pre-Calculus)}}
\subsection{General Trigonometric}
$cos^2\theta + sin^2\theta=1$\\
$\sec^2{x}=1+\tan^2{x}$\\
$\tan{\theta}=\dfrac{\sin{\theta}}{\cos{\theta}}$

\subsection{Double-Angle}
$cos(2\theta)=\cos^2{\theta}-\sin^2{\theta}$\\
$cos(2\theta)=1-2\sin^2{\theta}$\\
$cos(2\theta)=2\cos^2{\theta}-1$\\
$sin(2\theta)=2\cos{\theta}\cdot\sin{\theta}$\\
$\tan(2\theta)=\dfrac{2\tan{\theta}}{1-\tan^2{\theta}}$

\subsection{Half-Angle}
$\sin(\frac{\theta}{2})=\pm\sqrt{\dfrac{1-\cos{\theta}}{2}}$\\
$\cos(\frac{\theta}{2})=\pm\sqrt{\dfrac{1+\cos{\theta}}{2}}$\\
$\tan(\frac{\theta}{2})=\pm\sqrt{\dfrac{1-\cos{\theta}}{1+\cos{\theta}}}$\\[3pt]
$\tan(\frac{\theta}{2})= \dfrac{\sin{\theta}}{1+\cos{\theta}}$\\
$\tan(\frac{\theta}{2})= \dfrac{1-\cos{\theta}}{\sin{\theta}}$

\section{Algebra\footnotesize{(Pre-Calculus)}}
\subsection{Identités Remarquables}
\subsubsection{Second Degrées}
$(a+b)^2=a^2+2ab+b^2$\\
$(a-b)^2=a^2-2ab+b^2$\\
$a^2-b^2=(a+b)(a-b)$\\
$a^2+b^2=(a+ib)(a-ib)$

\subsubsection{Troisiéme Degrées}
$(a+b)^3 = a^3+3a^2b+3ab^2+b^3$\\
$(a-b)^3 = a^3-3a^2b+3ab^2-b^3$\\
$a^3+b^3 = (a+b)(a^2-2ab+b^2)$\\
$a^3-b^3 = (a-b)(a^2+2ab+b^2)$

\subsection{Logarithm}
$\log_b(a)= c\Leftrightarrow a=b^c$\\
$\log(a\cdot b)=\log{a} + \log{b}$\\
$\log(\frac{a}{b})=\log{a}-\log{b}$\\
$\log(a^b)=b\cdot\log(a)$\\
$\log_a{x}=\forall{n}\dfrac{\log_n{x}}{\log_n{a}}$

\section{Calculus}
\subsection{Limits}
\subsubsection{Properties}
\(\lim\limits_{x\to a} c\cdot f(x) = c\cdot \lim_{x\to a} f(x)\)\\
\(\lim\limits_{x\to 0} \dfrac{\sin(\alpha x)}{\alpha x}=1, \alpha\in\mathbb{R}\)

\subsubsection{Theorems}
\subsubsection{Limits Simplified Theorem}
\begin{subequations}
\begin{equation}
    \lim_{x\to a}f(x)=L
\end{equation}
\begin{equation}
    \Rightarrow\lim_{x\to a^+}f(x)=\lim_{x\to a^-}f(x)=L
\end{equation}
\end{subequations}

\subsubsection{Limits Formal Theorem}
\begin{subequations}
Let $\epsilon\in\mathbb{R}$ and $\delta\in\mathbb{R}$
\begin{equation}
    \lim_{x\to a}f(x)=L\Rightarrow\forall\epsilon\exists\delta(\epsilon>0\wedge\delta>0)
\end{equation}
such that:
\begin{equation}
    \forall{x}(\mid f(x)-L\mid<\epsilon\wedge\mid x-a\mid<\delta)
\end{equation}
\end{subequations}

\subsubsection{Squeeze Theorem}
\begin{subequations}
\begin{equation}
    g(x)\ge f(x)\ge h(x)
\end{equation}
if $\lim\limits_{x\to a}g(x) = \lim\limits_{x\to a}h(x) = L$ then:
\begin{equation}
    f(x) = L
\end{equation}
\end{subequations}
Let $I= [A,B]$ be an interval of $\mathbb{R}$\\
If $f(A)<u<f(B)$\\
Then $\exists c\in (A,B) \:|\: f(c)=u$


\subsubsection{L'hopital theorem}
Let $g(x)$ and $f(x)$ be some function differentiable on $x=a$.\\
If $(\lim\limits_{x\to a} f(x)=0\wedge\lim\limits_{x\to a} g(x)=0)\vee(\lim\limits_{x\to a} f(x)=\pm\infty\wedge\lim\limits_{x\to a} g(x)=\pm\infty)$ then:

\begin{center}
$\lim\limits_{x\to a}\dfrac{f(x)}{g(x)}\stackrel{\text{LH}}{=}\lim\limits_{x\to a}\dfrac{f'(x)}{g'(x)}$    
\end{center}






\subsubsection{Mean Value Theorem}
1) $f(x)$ is continuious on $[A,B]$\\
2) $f(x)$ is differentialble $(A,B)$\\
Then $\exists c\in (A,B)\: f'(c)=\dfrac{f(B)-f(A)}{B-A}$
\subsubsection{Rolle's Theorem}
1) $f(x)$ is continuious on $[A,B]$\\
2) $f(x)$ is differentialble $(A,B)$\\
3) $f(A)=f(B)$\\
Then $\exists c\in (A,B)\: f'(c)=0$

\subsubsection{Extreme value theorem}
If f is continuous on $[a,b]$ then f attain an abs max $f(c)$ and abs  min $f(d)$ for some c,d in $[a,b]$.
\begin{equation}
    f(d)\leq f(x)\leq f(c)
\end{equation}


\subsection{Derivative}
\subsubsection{Definition}
$\dfrac{d}{dx}f(x) = \lim\limits_{h\to0}\dfrac{f(x+h)-f(x)}{h}$
\subsubsection{Formulas}
\begin{tabular}{ll}
Constante & $\frac{d}{dx}a = 0$\\
Constante & $\frac{d}{dx}a\cdot f(x) = a\cdot\frac{d}{dx}f(x)$\\[1pt]
Sum & $\frac{d}{dx}f(x)+ g(x) = f'(x)+ g'(x)$\\[1pt]
Power & $\frac{d}{dx}a\cdot x^n = a\cdot n\cdot x^{n-1}$\\[1pt]
Square root & $\frac{d}{dx}\sqrt{x} = -\frac{1}{2\sqrt{x}}$\\[1pt]
Product & $\frac{d}{dx}f\cdot g = f'\cdot g + f\cdot g'$\\[1pt]
Quotient & $\frac{d}{dx}\frac{f}{g} = \dfrac{f'\cdot g - f\cdot g'}{g^2}$\\[1pt]
Logarithm & $\frac{d}{dx}\log_ax=\dfrac{1}{x\ln{a}}$\\[1pt]
Natural log & $\frac{d}{dx}\ln{x}=\dfrac{1}{x}$\\[1pt]
Chain & $\frac{d}{dx}f(g(x)) = f'(g(x))\cdot g'(x)$\\
\end{tabular}
\subsubsection{Trigonometric Formula}
\begin{tabular}{ll}
Sinus & $\frac{d}{dx}\sin{x} = \cos{x}$\\[1pt]
Cosinus & $\frac{d}{dx}\cos{x} = -\sin{x}$\\[1pt]
Tangent & $\frac{d}{dx}\tan{x} = \sec^2{x}$\\[1pt]
Cotangent & $\frac{d}{dx}\cot{x}=-\csc^2x$\\[1pt]
Second & $\frac{d}{dx}\sec{x}=\sec{x}\tan{x}$\\[1pt]
Cosecant & $\frac{d}{dx}\csc{x}=-\csc{x}\cot{x}$\\[1pt]

Arc Sinus & $\frac{d}{dx}\arcsin{x}=\dfrac{1}{\sqrt{1-x^2}}$\\[1pt]
Arc Cousins & $\frac{d}{dx}\arccos{x}=-\dfrac{1}{\sqrt{1-x^2}}$\\[1pt]
Arc Tangent & $\frac{d}{dx}\arctan{x}=\dfrac{1}{x^2+1}$\\[1pt]
Arc Cotangent & $\frac{d}{dx}arccot\, x=-\dfrac{1}{x^2+1}$\\[1pt]
Arc Second & $\frac{d}{dx}arcsec\,x=\dfrac{1}{|x|\sqrt{1-x^2}}$\\[1pt]
Arc Cosecant & $\frac{d}{dx}arccsc\,x=-\dfrac{1}{|x|\sqrt{1-x^2}}$\\[1pt]
\end{tabular}

\subsection{Linearization and differential}
\subsubsection{Linearization}
GOAL: Simplify relatively complicated function by approximation with a linear function.
$L(x)=F'(a)(x-a+)+F(a)$

\subsection{Sum}
\begin{tabular}{ll}
propertie 1 & $\sum^n_{i=1}a=an$\\
propertie 2 & $\sum^n_{i=1}ca=c\sum^n_{i=1}a$\\
Expantion 1 & $\sum^n_{i=1}i=\dfrac{n(n+1)}{2}$\\
Expantion 2 & $\sum^n_{i=1}i^2=\dfrac{n(n+1)(2n+1)}{6}$\\
Expantion 3 & $\sum^n_{i=1}i^3=\dfrac{n^2(n+1)^2}{4}$\\
 Rieman sum & $\int_a^bf(x)\;dx$\\
 & $=\lim\limits_{n\to\infty}\sum^n_{i=1}f(x_i)\Delta x$\\
 Where & $\Delta x = \frac{b-a}{n}$\\
 & $x_i=a+i\Delta x$
\end{tabular}

\subsection{Integrals}
\subsubsection{Properties of the Definite Integral}
\begin{tabular}{ll}
$\int^a_af(x)\:dx=0$ & \\
$\int^a_bf(x)\:dx=-\int^b_af(x)\:dx$ & \\
$\int^a_bcf(x)\:dx=c\int^a_bf(x)\:dx$, where c is a const. & \\
$\int^c_af(x)\:dx+\int^b_cf(x)\:dx=\int^b_af(x)\:dx$& \\
\end{tabular}

\subsubsection{Even-Odd Symmetry}
Suppose $f$ is continuous on $[-a,a]$.\\
\begin{tabular}{ll}
If $f$ is even & $\int^a_{-a}f(x)\:dx=2\int^a_0f(x)\:dx$\\
If $f$ is odd & $\int^a_{-a}f(x)\:dx=0$\\
\end{tabular}

\subsubsection{Fundamental Thm Of Calculus}
If $f$ is cont on $[a,b]$ then\\
Part: 1\\
\begin{equation}
    f(x)=\dfrac{d}{dx}\int^x_af(t)\: dt
\end{equation}
Where $a\leq x\leq b, a\leq t\leq x$. Then\\
Part: 2\\
\begin{equation}
\int^b_af(x)\:dx=F(x)\bigg|_a^b=F(b)-F(a)
\end{equation}

\subsection{Hyperbolic functions}
\subsubsection{Definition}
$\cosh{x} = \dfrac{e^x + e^{-x}}{2}$\\
$\sinh{x} = \dfrac{e^x - e^{-x}}{2}$\\
$\tanh{x} = \dfrac{\sinh{x}}{\cosh{x}} = \dfrac{e^x - e^{-x}}{e^x + e^{-x}} $\\
\subsubsection{Derivative}
$\frac{d}{dx}\cosh{x}=\sinh{x}$\\
$\frac{d}{dx}\sinh{x}=\cosh{x}$\\
$\frac{d}{dx}\tanh{x}=sech^2\,x$\\
$\frac{d}{dx}\coth{x}=-csch^2\,x$\\
$\frac{d}{dx}csch\,x=-csch\,x\coth{x}$\\
$\frac{d}{dx}sech\,x=sech\,x\coth{x}$\\

\end{multicols}
\end{document}