\documentclass[10pt]{article}
\usepackage{multicol,multirow}
\usepackage{graphicx} % Required for inserting images
\usepackage[margin=0.45cm]{geometry}
\usepackage{xcolor}
\usepackage{array}

\definecolor{LightGray}{gray}{0.9}

\usepackage{minted}

\begin{document}
\begin{center}
  \begin{picture}(0,0)
    \put(-60, -7){\includegraphics[width=2cm]{Git-Logo.png}} % adjust coordinates here
  \end{picture}
    \Large{\textbf{Cheat Sheet}}\\
     \hfill\small{\textcopyright Maximilien Notz \the\year{}}
     \noindent\rule{20cm}{0.4pt}
\end{center}
\begin{multicols}{2}
\setcounter{secnumdepth}{0}

\subsection{Terminal Commands}

All the following commands are to be run in a terminal, the keywords \texttt{git} must be prefixed to each of them.
\begin{tabular}{>{\ttfamily}l l}
    clone url           & Clone a repository from \texttt{url} into a new\\
                        & directory.\\
    status              & Display the state of the working directory.\\
    log                 & Show all commit logs of the current\\
                        & repository.\\
    add file            & Add \texttt{file} to the staging area.\\
    add -A              & Add all changes to the staging area. \\
    commit -m "msg"     & Commit the staged changes with message \\
                        & \texttt{msgg}. \\
    reset               & delete all local changes. \\
    push origin branch  & Push the commits to remote \texttt{origin} on \\
                        & \texttt{branch}. \\
    push                & Push the commits to the default remote \\
                        & and branch. \\
    rm fileName         & remove a file from the repo.\\
\end{tabular}

\subsection{GitHub}

\subsection{Open-Source Licenses}
\begin{tabular}{>{\ttfamily}l l}
  Copyleft Strength      & Determines how strongly the license requires\\
              & derivative works to use the same license.\\
  Commercial Use        & Whether the code can be used in commercial\\
              & products or services.\\
  Attribution          & Requirements for giving credit to original\\
              & authors and maintaining notices.\\
  Patent Protection    & Whether the license includes explicit\\
              & protection against patent claims.\\
  Work Flexibility     & Conditions and restrictions on how the\\
              & code can be modified and redistributed.\\
  Typical Use Cases    & Common projects and applications where\\
              & the license is frequently used.\\
\end{tabular}
\end{multicols}
\begin{tabular}{|>{\small}p{2.1cm}|>{\small}p{1.8cm}|>{\small}p{2.3cm}|>{\small}p{2.1cm}|>{\small}p{1.8cm}|>{\small}p{2.5cm}|>{\small}p{5cm}|}
\hline
\textbf{License} & \textbf{Copyleft Strength} & \textbf{Commercial Use Allowed} & \textbf{Attribution Required} & \textbf{Patent Protection} & \textbf{Derivative Work Flexibility} & \textbf{Typical Use Cases / Examples} \\
\hline
MIT           & None (Permissive) & Yes & Yes (license \& copyright) & No explicit & Unlimited (incl.\ proprietary) & Libraries, web frameworks, commercial reuse (React, jQuery, Rails). \\
\hline
Apache 2.0    & None (Permissive) & Yes & Yes (NOTICE \& license) & Explicit patent grant & Unlimited (incl.\ proprietary) & Enterprise/cloud, ML frameworks (Android, TensorFlow, Apache Spark). \\
\hline
GNU GPL v3    & Strong (Copyleft) & Yes\footnotesize~(derivatives must be GPL when distributed) & Yes (license \& source terms) & Explicit contributor patent license & Must remain GPL (project-wide) & GNU/Linux ecosystem, dev tools (GCC, Bash, GIMP). \\
\hline
Mozilla Public License 2.0 & Weak (File-based) & Yes & Yes (license/NOTICE) & Explicit patent grant & File-based copyleft & Mixed open/proprietary codebases (Firefox, Thunderbird). \\
\hline
\end{tabular}

\end{document}
