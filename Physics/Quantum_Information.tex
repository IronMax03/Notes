\documentclass[5pt]{article}
\usepackage{multicol,multirow}
\usepackage{graphicx} % Required for inserting images
\usepackage[margin=0.75cm]{geometry}
\usepackage{xcolor}
\usepackage{amsmath,esint}
\usepackage{mathtools}
\usepackage{relsize}
\usepackage{mathtools}
\usepackage{nccmath}
\usepackage[inline]{enumitem}
\usepackage{algpseudocode}
\usepackage{physics}

\usepackage{empheq}
\usepackage{amsfonts}

\definecolor{LightGray}{gray}{0.9}

\usepackage{minted}


\makeatletter
\let\oldabs\abs
\def\abs{\@ifstar{\oldabs}{\oldabs*}}


\begin{document}

\newtheorem{theorem}{Theorem}
\newtheorem{properties}{Properties}

\begin{center}
     \Large{\textbf{Quantum Information}}\\
     \small{Class: IBM Quantum Learning}\hfill\small{\textcopyright Maximilien Notz \the\year{}}
     \noindent\rule{20.2cm}{0.4pt}
\end{center}


\begin{multicols}{3}
\setcounter{secnumdepth}{0}

\subsection{Linear Algebra}

\begin{tabular}{ll}
     Complexe conjugate  & $(a+ib)^*=a-ib$\\
                         & $(a-ib)^*=a+ib$\\
                         & where $a\in\mathbb{R}$, $b\in\mathbb{R}$\\
\end{tabular}


\begin{properties}[Complexe conjugate]
     \begin{itemize*}
          \item $(Z^*)^*=Z$
          \item $(Z+W)^*=Z^*+W^*$
          \item $(Z-W)^*=Z^*-W^*$
          \item $(ZW)^*=Z^*W^*$
          \item $Z^*Z=\abs{Z}^2$
          \item $(Z^n)^*=(Z^*)^n$, for $n\in\mathbb{Z}$
          \item $ln(Z^*)=(ln(Z))^*$ if $Z$ is not 0 or a negative  real number.
     \end{itemize*}
\end{properties}

\subsubsection{Linear Operator}
A \textbf{linear operator} between the vector spaces $V$ and $W$ is define to be any function $A:V\rightarrow W$ which is linear in its input.
\begin{properties}
     Let $\hat{A}$ be a linear operator on $V\rightarrow W$  and $A$ be the matrix representation of $\hat{A}$.
     \begin{itemize*}
          \item $\hat{A}(\sum_ia_i\ket{v_i})=\sum_ia_i\hat{A}\ket{v_i}$\\
          \item $\hat{A}\ket{v_j}=\sum_iA_{ij}\ket{w_i}$\\
     \end{itemize*}
\end{properties}


\subsubsection{Inner product}





\subsection{Quantum mecanics}
\subsubsection{Pauli Matrices}
\begin{itemize*}
     \item $\sigma_0 = I = \begin{bmatrix}1 & 0 \\ 0 & 1\end{bmatrix}$
     \item $\sigma_x = X = \begin{bmatrix}0 & 1 \\ 1 & 0\end{bmatrix}$
     \item $\sigma_y = Y = \begin{bmatrix}0 & -i \\ i & 0\end{bmatrix}$
     \item $\sigma_z = Z = \begin{bmatrix}1 & 0 \\ 0 & -1\end{bmatrix}$
\end{itemize*}




\newpage





\subsection{Mathematical Concepts}

\subsubsection{Dirac Notation\footnotesize{(or Bra-Ket Notation)}}
Terminology: ket of $A$ is $\ket{A}$ and bra of $A$ is $\bra{A}$.
Example, let $\Sigma=\{A,B,C\}$ then $\ket{A}=\begin{bmatrix}1\\0\\0 \end{bmatrix}$, $\ket{B}=\begin{bmatrix}0\\1\\0 \end{bmatrix}$
and $\bra{A}=(1,0,0)$, $\bra{B}=(0,1,0)$
Note that $\ket{\Psi}$ then $\bra{\Psi}=\ket{\Psi}^T$.\\
\textbf{bra-kets:} we denote $\braket{a}{b}$ the matrice product of 


\subsubsection{Cartesian product}
Let $y=\{0,1\}$ and  $x=\{a,b\}$. Then, \\
$x\times y = \{(0,a),(0,b),(1,a),(1,b)\}$ and \\
$y\times x = \{(a,0),(a,1),(b,0),(b,1)\}$.

\subsubsection{Tensor Product of vectors}
$\begin{bmatrix}a_0\\ \vdots\\a_n \end{bmatrix}\otimes\begin{bmatrix}b_0\\ \vdots\\b_n\ \end{bmatrix}= \begin{bmatrix}a_0b_0\\ \vdots\\a_nb_n\\ \end{bmatrix}$


\subsubsection{Tensor Product Properties}
1. $(\ket{\phi_1}+\ket{\phi_2})\otimes\ket{\psi}=\ket{\phi_1}\otimes\ket{\psi}+\ket{\phi_2}\otimes\ket{\psi}$\\
2. $(a\ket{\phi})\otimes\ket{\psi}=a(\ket{\phi}\otimes\ket{\psi})$\\
3. $\ket{a}\otimes\ket{b}=\ket{b}\otimes\ket{a}$

\subsection{Quantum Information Systems}
\subsection{State Vector}
Quantum State of a sytem is represented by a complex column vector.
Let the quantum state vector $v$ be equal to $\begin{bmatrix}a_0\\ \vdots\\a_n\\ \end{bmatrix}$, where $\sum^n_{i=0}\abs{a_i}^2=1$
The \textbf{euclidean norm} of the $||v||=\sqrt{\sum^n_{i=0}\abs{a_i}^2}$

\subsubsection{Common Quantum States}
\begin{tabular}{ll}
     Plus State     & $\ket{+}=\frac{1}{\sqrt{2}}\ket{0}+\frac{1}{\sqrt{2}}\ket{1}$ \\
     Minus State    & $\ket{-}=\frac{1}{\sqrt{2}}\ket{0}-\frac{1}{\sqrt{2}}\ket{1}$ \\
     Other State    & $\frac{1+2i}{3}\ket{0}-\frac{2}{3}\ket{1}$
\end{tabular}

\subsection{Standart Basis Measurement}
Let a quantum system be in the state $\ket{\psi}$, 
then the probability for the measure outcome to be $a$ is $Pr(\text{outcome} = a)=\abs{\braket{a}{\psi}}^2$
If $U$ is an unary matrice then the following Propertie hold, $||U\psi||=||\psi||$ 

\subsection{Unary Operations}
\subsubsection{Unary Matrice}
A squared matrix $U$ having complex number entries is unitary if it satisfies the equations, 
$UU^\dagger=U^\dagger U=\mathbb{I}$ where $\mathbb{I}$ is the identity matrix.

\subsubsection{Some unitary operations on qubits}
Pauli operations:\\
$\mathbb{I}=\begin{pmatrix}
     1 & 0 \\
     0 & 1 \\
\end{pmatrix}$ 
$\sigma_x =\begin{pmatrix}
     0 & 1 \\
     1 & 0 \\
\end{pmatrix}$
$\sigma_y =\begin{pmatrix}
     0 & -i \\
     i & 0 \\
\end{pmatrix}$
$\sigma_z =\begin{pmatrix}
    -1 & 0 \\
     0 & 1 \\
\end{pmatrix}$ \\
Hadamard operation: 
$H =\begin{pmatrix}
     \frac{1}{\sqrt{2}} & 0 \\
     0 & \frac{1}{\sqrt{2}} \\
\end{pmatrix}$ \\
Phase operations:
$P_\theta =\begin{pmatrix}
     1 & 0 \\
     0 & e^{i\theta} \\
\end{pmatrix}$ \\



\subsection{Quantum cirtcuit}

\end{multicols}
\end{document}
