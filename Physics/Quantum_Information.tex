\documentclass[5pt]{article}
\usepackage{multicol,multirow}
\usepackage{graphicx} % Required for inserting images
\usepackage[margin=0.75cm]{geometry}
\usepackage{xcolor}
\usepackage{amsmath,esint}
\usepackage{mathtools}
\usepackage{relsize}
\usepackage{mathtools}
\usepackage{nccmath}
\usepackage[inline]{enumitem}
\usepackage{algpseudocode}
\usepackage{physics}
\usepackage[frak=esstix]{mathalpha}

\usepackage{empheq}
\usepackage{amsfonts}

\definecolor{LightGray}{gray}{0.9}

\usepackage{minted}


\makeatletter
\let\oldabs\abs
\def\abs{\@ifstar{\oldabs}{\oldabs*}}


\begin{document}

\newtheorem{theorem}{Theorem}
\newtheorem{properties}{Properties}

\begin{center}
     \Large{\textbf{Quantum Computation \& Quantum Information}}\\
     \small{Book: Quantum Computation and Quantum Information.  Nielsen, M. A., \& Chuang, I. L. }\hfill\small{\textcopyleft Maximilien Notz \the\year{}}
     \noindent\rule{20.2cm}{0.4pt}
\end{center}


\begin{multicols}{3}
\setcounter{secnumdepth}{0}


\subsection{Complex Analysis Short recap}
Let $a,b\in\mathbb{R}$ and $z\in\mathbb{C}$ such that $z=a+bi$.
\begin{tabular}{ll}
     Real part           & $\mathfrak{Re}(a+ib)=a$\\
     Real part           & $\mathfrak{Im}(a+ib)=b$\\
     Absolute Values     & $|z|=\sqrt{zz^*}$\\
                         & $= \norm{(a\;\:b)}_2$\\
     Complexe conjugate  & $(a+ib)^*=a-ib$\\
                         & $(a-ib)^*=a+ib$\\
     Trig. Formulas      & $\sin{z}=\frac{e^{iz}-e^{-iz}}{2i}$\\
     Trig. Formulas      & $\cos{z}=\frac{e^{iz}+e^{-iz}}{2}$\\
\end{tabular}


\begin{properties}[Complexe conjugate]
     \begin{itemize*}
          \item $(Z^*)^*=Z$
          \item $(Z+W)^*=Z^*+W^*$
          \item $(Z-W)^*=Z^*-W^*$
          \item $(ZW)^*=Z^*W^*$
          \item $Z^*Z=\abs{Z}^2$
          \item $(Z^n)^*=(Z^*)^n$, for $n\in\mathbb{Z}$
          \item $ln(Z^*)=(ln(Z))^*$ if $Z$ is not 0 or a negative  real number.
     \end{itemize*}
\end{properties}


\begin{properties}[Absolute Values]
     \begin{itemize*}
          \item $|z_1z_2|=|z_1|\:|z_2|$
          \item The absolute value define the metric of the space $\mathbb{C}$ ($\mathbb{C}$ is complete).
          \item $|z_1+z_2|^2=|z_1|^2+|z_2|^2+\mathfrak{Re}(z_1z_2^*)$
          \item $|z_1-z_2|^2=|z_1|^2+|z_2|^2-\mathfrak{Re}(z_1z_2^*)$
          \item $|z_1+z_2|^2+|z_1-z_2|^2=2(|z_1|^2+|z_2|^2)$
     \end{itemize*}
\end{properties}



\subsection{Linear Algebra}

\subsubsection{Linear Operator}
A \textbf{linear operator} between the vector spaces $V$ and $W$ is define to be any function $\hat{A}:V\rightarrow W$ which satisfies:\\ $\hat{A}(\alpha\vec{v}+\beta\vec{w})=\alpha\hat{A}\vec{v}+\beta\hat{A}\vec{w}$.

\begin{properties}
     Let $\hat{A}$ be a linear operator on $V\rightarrow W$  and $A$ be the matrix representation of $\hat{A}$.
     \begin{itemize*}
          \item $\hat{A}(\sum_ia_i\ket{v_i})=\sum_ia_i\hat{A}\ket{v_i}$\\
          \item $\hat{A}\ket{v_j}=\sum_iA_{ij}\ket{w_i}$\\
     \end{itemize*}
\end{properties}


\subsubsection{Inner product}
A Inner Product $\langle.,.\rangle$ is a function that output a complex number and satisfies the following conditions: Let $\vec{v}\in\mathbb{C}^n, \vec{w}\in\mathbb{C}^n$.
\begin{enumerate}
     \item $\langle \vec{v},\sum_ia_i\vec{w_i}\rangle=\sum_ia_i\langle \vec{v},\vec{w_i}\rangle$
     \item $\langle \vec{v},\vec{w}\rangle=(\langle \vec{w},\vec{v}\rangle)^*$
     \item if $x=0$ and only if $\langle \vec{w},\vec{w}\rangle\geq 0$
\end{enumerate}
In quantum mecanics the inner product is generaly noted $\langle.|.\rangle$.

\begin{properties}
     \begin{itemize*}
          \item $\langle A,A\rangle=\norm{A}^2$
          \item if $\langle A,B\rangle=0$ then $A$ and $B$ are orthogonal.      
     \end{itemize*}
\end{properties}

\subsubsection{Inner product Space}

An inner product space is a vector space $V$ equipped with an inner product $\langle\cdot,\cdot\rangle : V \times V \to \mathbb{C}$ (or $\mathbb{R}$).

An Inner product space with an orthonormal basis $\ket{i}$ such that $v=\sum_iv_i\ket{i}$ and $w=\sum_iw_i\ket{i}$, the inner product space of $\langle v,w \rangle$ is define by $(v^*)^Tw$.

 
\subsubsection{Hilbert Spaces}
A Hilbert space is a vector space(generaly complex) equipped with an inner product, 
meaning every Cauchy sequence of vectors with respect to the induced norm converges to a vector within the space.
\textbf{In finit dimentions hilbert spaces is exactly the same thing as Inner Product space.}


\subsubsection{Dirac Notation \footnotesize{(or Bra-Ket Notation)}}
Terminology: ket of $A$ is $\ket{A}$ and bra of $A$ is $\bra{A}$. ket is a row vector and bra is a column vector.\\
Example, let $\Sigma=\{A,B,C\}$ then\\
$\ket{B}=\begin{bmatrix}0\\1\\0 \end{bmatrix}$ and $\bra{B}=(0,1,0)$\\

\subsubsection{Kronecker delta($\delta_{ij}$)}
\begin{equation*}
    \delta_{ij} =
    \begin{cases}
        1 & \text{if } i = j, \\
        0 & \text{if } i \neq j.
    \end{cases}
\end{equation*}
\begin{properties}
     \begin{itemize*}
          \item $\langle i|j\rangle=\delta_{ij}$
          \item $\mathbb{I}_{ij}=\delta_{ij}$
          \item $\sum_i\delta_{ij}a_i=a_j$ 
          \item $\sum_k\delta_{ik}\delta_{kj}=\delta_{ij}$ 
     \end{itemize*}
\end{properties}

\subsubsection{Gram-Schmidt\footnotesize{(in Dirac Notation)}}
\begin{equation*}
     \ket{v_1}=\frac{\ket{w_1}}{\norm{\ket{w_1}}}\\
\end{equation*}
\begin{equation*}
     \ket{v_{k+1}}=\frac{\ket{w_{k+1}}-\sum_{i=1}^k\braket{v_i}{w_{k+1}}\ket{v_i}}{\norm{\ket{w_{k+1}}-\sum_{i=1}^k\braket{v_i}{w_{k+1}}\ket{v_i}}}
\end{equation*}

\subsubsection{Adjoin \small{(Hermician conjugate)}}
Let $\hat{A}$ be a linear operator on the hilbert space $V$. 
$\exists! \hat{A}^\dagger$ on $V$ such that $\langle v, \hat{A}w\rangle=\langle\hat{A}^\dagger v, w\rangle$.

\begin{properties}
     \begin{itemize*}
          \item $\ket{v}^\dag=\bra{v}\;$
          \item $(\hat{A}^\dag)^\dag=\hat{A}$
          \item $(\hat{A}\hat{B})^\dagger=\hat{A}^\dagger\hat{B}^\dagger\;\:$     
          \item $(\sum_ia_i\hat{A}_i)^\dag=\sum_i a_i^*\hat{A}_i^\dag$
     \end{itemize*}
\end{properties}


\subsection{Quantum mecanics}
\subsubsection{Pauli Matrices}
\begin{itemize*}
     \item $\sigma_0 = I = \begin{bmatrix}1 & 0 \\ 0 & 1\end{bmatrix}$
     \item $\sigma_x = X = \begin{bmatrix}0 & 1 \\ 1 & 0\end{bmatrix}$
     \item $\sigma_y = Y = \begin{bmatrix}0 & -i \\ i & 0\end{bmatrix}$
     \item $\sigma_z = Z = \begin{bmatrix}1 & 0 \\ 0 & -1\end{bmatrix}$
\end{itemize*}









\newpage

\subsection{Mathematical Concepts}

\subsubsection{Dirac Notation\footnotesize{(or Bra-Ket Notation)}}
Terminology: ket of $A$ is $\ket{A}$ and bra of $A$ is $\bra{A}$.
Example, let $\Sigma=\{A,B,C\}$ then $\ket{A}=\begin{bmatrix}1\\0\\0 \end{bmatrix}$, $\ket{B}=\begin{bmatrix}0\\1\\0 \end{bmatrix}$
and $\bra{A}=(1,0,0)$, $\bra{B}=(0,1,0)$
Note that $\ket{\Psi}$ then $\bra{\Psi}=\ket{\Psi}^T$.\\
\textbf{bra-kets:} we denote $\braket{a}{b}$ the matrice product of 


\subsubsection{Cartesian product}
Let $y=\{0,1\}$ and  $x=\{a,b\}$. Then, \\
$x\times y = \{(0,a),(0,b),(1,a),(1,b)\}$ and \\
$y\times x = \{(a,0),(a,1),(b,0),(b,1)\}$.

\subsubsection{Tensor Product of vectors}
$\begin{bmatrix}a_0\\ \vdots\\a_n \end{bmatrix}\otimes\begin{bmatrix}b_0\\ \vdots\\b_n\ \end{bmatrix}= \begin{bmatrix}a_0b_0\\ \vdots\\a_nb_n\\ \end{bmatrix}$


\subsubsection{Tensor Product Properties}
1. $(\ket{\phi_1}+\ket{\phi_2})\otimes\ket{\psi}=\ket{\phi_1}\otimes\ket{\psi}+\ket{\phi_2}\otimes\ket{\psi}$\\
2. $(a\ket{\phi})\otimes\ket{\psi}=a(\ket{\phi}\otimes\ket{\psi})$\\
3. $\ket{a}\otimes\ket{b}=\ket{b}\otimes\ket{a}$

\subsection{Quantum Information Systems}
\subsection{State Vector}
Quantum State of a sytem is represented by a complex column vector.
Let the quantum state vector $v$ be equal to $\begin{bmatrix}a_0\\ \vdots\\a_n\\ \end{bmatrix}$, where $\sum^n_{i=0}\abs{a_i}^2=1$
The \textbf{euclidean norm} of the $||v||=\sqrt{\sum^n_{i=0}\abs{a_i}^2}$

\subsubsection{Common Quantum States}
\begin{tabular}{ll}
     Plus State     & $\ket{+}=\frac{1}{\sqrt{2}}\ket{0}+\frac{1}{\sqrt{2}}\ket{1}$ \\
     Minus State    & $\ket{-}=\frac{1}{\sqrt{2}}\ket{0}-\frac{1}{\sqrt{2}}\ket{1}$ \\
     Other State    & $\frac{1+2i}{3}\ket{0}-\frac{2}{3}\ket{1}$
\end{tabular}

\subsection{Standart Basis Measurement}
Let a quantum system be in the state $\ket{\psi}$, 
then the probability for the measure outcome to be $a$ is $Pr(\text{outcome} = a)=\abs{\braket{a}{\psi}}^2$
If $U$ is an unary matrice then the following Propertie hold, $||U\psi||=||\psi||$ 

\subsection{Unary Operations}
\subsubsection{Unary Matrice}
A squared matrix $U$ having complex number entries is unitary if it satisfies the equations, 
$UU^\dagger=U^\dagger U=\mathbb{I}$ where $\mathbb{I}$ is the identity matrix.

\subsubsection{Some unitary operations on qubits}
Pauli operations:\\
$\mathbb{I}=\begin{pmatrix}
     1 & 0 \\
     0 & 1 \\
\end{pmatrix}$ 
$\sigma_x =\begin{pmatrix}
     0 & 1 \\
     1 & 0 \\
\end{pmatrix}$
$\sigma_y =\begin{pmatrix}
     0 & -i \\
     i & 0 \\
\end{pmatrix}$
$\sigma_z =\begin{pmatrix}
    -1 & 0 \\
     0 & 1 \\
\end{pmatrix}$ \\
Hadamard operation: 
$H =\begin{pmatrix}
     \frac{1}{\sqrt{2}} & 0 \\
     0 & \frac{1}{\sqrt{2}} \\
\end{pmatrix}$ \\
Phase operations:
$P_\theta =\begin{pmatrix}
     1 & 0 \\
     0 & e^{i\theta} \\
\end{pmatrix}$ \\



\subsection{Quantum cirtcuit}

\end{multicols}
\end{document}
