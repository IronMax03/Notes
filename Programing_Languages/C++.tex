\documentclass[5pt]{article}
\usepackage{multicol,multirow}
\usepackage{graphicx} % Required for inserting images
\usepackage[margin=0.45cm]{geometry}
\usepackage{xcolor}
\usepackage{array}

\definecolor{LightGray}{gray}{0.9}

\usepackage{minted}

\begin{document}
\begin{center}
     \Large{\textbf{C\texttt{++} Cheat Sheet}}\\
     \small{Class: CSCI 1300}\hfill\small{\textcopyright Maximilien Notz \the\year{}}
     \noindent\rule{20cm}{0.4pt}
\end{center}
\begin{multicols}{2}
\setcounter{secnumdepth}{0}


\subsection{General Reminders}
\begin{tabular}{>{\ttfamily}l l}

\#include "myfile.h"    & Include file.\\
rand()                  & Random int, \texttt{\#include<cstdlib>}.\\
int(var)                & Convert a data type var to int.\\
float(var)              & Convert a data type var to float.\\
double(var)             & Convert a data type var to double.\\
static\_cast<t>(var)    & Convert \texttt{var} to the type \texttt{t}.\\
void myF() const        & read only fonction.\\
inline                  & the whole code of the inline function is\\
                        & inserted or substituted at the point of its\\
                        & call during the compilation.\\
sizeof(var)             & return the number of bytes used by the\\
                        & variable. \texttt{sizeof} runs at compile-time.\\
ternary Opearator       & condition ? ifTrue : ifFalse\\
\end{tabular}


\subsection{Strings}
\begin{tabular}{>{\ttfamily}l l}
str$[$i$]$          & Get or set the char at the index \texttt{i}.\\
str.length()        & Return the number of characters.\\
str.substr(a,b)     & Returns the substring from a to b.\\
str.find(subStr)    & Retrun the start index of the substring\\
str.replace(i,l,str)& Replace substring from \texttt{i} to \texttt{l} with \texttt{str}\\
stoi(str)           & Convert a string to int,\\
                    & \texttt{\#include<string>}.\\
to\_string(var)     & Convert var to a string,\\
                    & \texttt{\#include<string>}.\\
\end{tabular}

\subsection{Arrays}
\begin{tabular}{c|c|c|c }
 0 & 1 & … & n \\ 
 \hline
 "Max" & "Tom" & … & arr$[$n$]$
\end{tabular}
This table illustrate the structure of an array of strings. Considering that n is equal to the number of element minus one. Arrays are a static data type.\\[6pt]
\begin{tabular}{>{\ttfamily}l l}
int arr[4];         & Create a array of int and with 4 element.\\
int arr[4]=\{6,3\}; & \\
arr$[$i$]$          & Get or set the element at the index i.\\
\end{tabular}


\subsection{Vectors}
\begin{tabular}{>{\ttfamily}l l}
\#include<vector>           & Include vector library.\\
vector<type> V;             & Instantiate a vector.\\
vector<type> V(size);       & Instantiate a vector from Array \texttt{obj}.\\
vector<type> V\{6,3,3\};    & Instantiate a vector from Array.\\
V=vector<type>();           & Re-instantiate \texttt{V}.\\
V.at(i)                     & Returns the element at index \texttt{i}.\\
V.size()                    & Return the number of elements.\\
V.push\_back(Value)         & Add the new element at the end.\\
V.pop\_back()               & Remove the last element.\\
V.clear()                   & Empty the vector.\\
V.insert(i, Value)          & Insert \texttt{Value} at \texttt{i}.
\end{tabular}


\subsection{Structures}
\begin{minted}[bgcolor=LightGray]{C++}
struct myStruct {
    string param1;    // atribute 1
    double param2;   // atribute 2
}s1, s2;            // myStruct instances
\end{minted}
\begin{tabular}{>{\ttfamily}l l}
myStruct Obj;   & instantiate structure object.\\
Obj.param1      & Access param1 of \texttt{Obj}.\\
\end{tabular}


\subsection{Streams}
\begin{tabular}{>{\ttfamily}l l}

\#include<fstream>      & Include stream library.\\
\#include<sstream>      & Include string stream library.\\
ifstream fin;           & Instantiate a input stream.\\
ofstream fout;          & Instantiate a output stream.\\
stringstream s(str);  & Instantiate a string stream.\\
myS.open("file.txt")    & Open txt file whith the stream.\\
myS.close()             & Close the stream file.\\
getline(fin, line)      & Get the next line from fin.\\
fout<<"hello"           & Output in stream "helloWorld".\\
fin>>var                & Input from stream to var.\\
%% streams manipulators
<<setprecision(n)<<     & Set decimal points, \texttt{\#include<iomanip>}.\\
<<setw(n)<<             & Establishes a print field of n spaces.\\
<<fixed<<               & Display floating point numbers in fixed.\\
                        & point notation.\\
<<showpoint<<           & Enables or disables the unconditional\\
<<noshowpoint<<         & inclusion of the decimal point character\\
                        & in floating-point output.\\
<<left<<                & output the string on the left.\\
<<right<<               & output the string on the right.\\
\end{tabular}

\subsubsection{clear buffer}
The buffer must be cleared after after getting an input from a stream if you input and output in the same file at the same time. 
\begin{minted}[bgcolor=LightGray]
{c++}
if(cin.fail() == true) {
    cout << "cin failed state";
    cin.clear(); 
    cin.ignore(1000, '\n'); 
}
\end{minted}


\subsection{cmath}
\begin{tabular}{>{\ttfamily}l l}
\#include<cmath>        & Include cmath library.\\
sqrt(x)                 & Square root of \texttt{x}.\\
pow(x, y)               & \texttt{x} raised to the power \texttt{y}.\\
abs(x)                  & Absolute value overloads.\\
floor(x)                & Greatest integer $\leq$ \texttt{x}.\\
ceil(x)                 & Smallest integer $\geq$ \texttt{x}.\\
fmod(x, y)              & Floating-point remainder of \texttt{x}$/$\texttt{y}.\\
\end{tabular}



\subsection{Error Handling}
\begin{minted}[bgcolor=LightGray]{C++}
try {
    // risky operation
} catch (exceptions) {
    // runs if an exception of type Ex is thrown
}
\end{minted}
\begin{tabular}{>{\ttfamily}l l}
\#include<cassert>          & Include assert library.\\
\#include<stdexcept>        & Common standard exceptions.\\
throw myException           & Throw an error of type myException.\\
exception::what()           & Retrieve diagnostic message.\\
catch (const auto\& e)      & Catch exceptions by const reference.\\
catch(...)                  & Fallback handler; rethrow if unsure.\\
exception                   & Parent of all exceptions class.\\
\end{tabular}

\subsection{Object Oriented Programing(OOP)}
\begin{minted}[bgcolor=LightGray]
{c++}
class myClasses :public parentClass{
    private:
        // private methods and variables
    public:
        // public methods and variables

        myClasses(int p1, int p2){ 
            // Body of constructor
        }

        ~myClasses(){
            // Body of destructor
        }
};
\end{minted}
\begin{tabular}{>{\ttfamily}l l}

myClasses myObj(3,5);   & Instantiate an myClasses type obj.\\
myClasses myObj;        & Call the default constructor.\\
protected:              & similar to private, but it can also be\\
                        & accessed in the inherited class.\\
\end{tabular}

\subsection{OOP With header file}
If you use a header the file wich contain the main function must include the header file.
\subsubsection{Header file(myHeader.h)}

\begin{minted}[bgcolor=LightGray]
{c++}
#ifndef MYCLASS_H //if no def for MyClass
#define MYCLASS_H //else

using namespace std;

class MyClass{
    public:
        MyClass(); //default constructor 
        MyClass(p1, p2); //parameterized constructor
        int publicAtribute;
        void myFunction() const;
    private:
        int privAtribute;
};
#endif
\end{minted}
\subsubsection{Class file(.cpp)}
\begin{minted}[bgcolor=LightGray]{c++}
#include<iostream>
#include "myHeader.h"

MyClass::MyClass(){
    publicAtribute = 0;
    privAtribute = 0;
}

MyClass::MyClass(int p1, int p2){
    publicAtribute = p1;
    privAtribute = p2;
}

void MyClass::myFunction() const{
    // my code
}
\end{minted}

\subsection{Switch case}
\begin{minted}[escapeinside=§§, mathescape=true, bgcolor=LightGray]
{c++}
switch (x){
    case 0:
        /*Code in case 0*/
    break;
    §$\vdots$§
    case n:
        /*Code in case n*/
    break;
    default:
        /*Code if no case match*/
}
\end{minted}

\subsection{Pointer \& References}
\begin{tabular}{>{\ttfamily}l l}    
    int* myInt;             & * means myInt work form a pointer.\\
    new                     & dynamically allocate a block of memory.\\
    delete                  & release dynamically allocated memory.\\
    NULL                    & Macro that referens to null pointer.\\
    $*$var                  & Get var value, where var is a pointer.\\
    \&var                   & Get memory addresse of \textbf{var}.\\
    void* var               & Pointer with no associated data type.\\
\end{tabular}

\subsection{Lambda Expression}
\begin{minted}[escapeinside=§§, mathescape=true, bgcolor=LightGray]{c++}
    ... = [captureClause] (parameters) -> returnType { 
    // definition
    }
\end{minted}
\begin{tabular}{>{\ttfamily}l l}
    $[\&]$                  & capture all external variables by reference.\\
    $[=]$                   & capture all external variables by value.\\
    $[a, \&b]$              & capture 'a' by value and 'b' by reference.\\
\end{tabular}

\subsection{Important ASCII Conversions}
\begin{center}
\begin{tabular}{|ll|ll|ll|ll|ll|}
\hline
\tiny{\emph{ASCII}} &  \emph{int} & \tiny{\emph{ASCII}} &  \emph{int} &  \tiny{\emph{ASCII}} &  \emph{int} &  \tiny{\emph{ASCII}} &  \emph{int}&  \tiny{\emph{ASCII}} &  \emph{int}\\
 \hline
A & 65 & a & 97  & N & 78 & n & 110 & 0 & 48 \\
B & 66 & b & 98  & O & 79 & o & 111 & 1 & 49 \\
C & 67 & c & 99  & P & 80 & p & 112 & 2 & 50 \\
D & 68 & d & 100 & Q & 81 & q & 113 & 3 & 51 \\
E & 69 & e & 101 & R & 82 & n & 114 & 4 & 52 \\
F & 70 & f & 102 & S & 83 & s & 115 & 5 & 53 \\
G & 71 & g & 103 & T & 84 & t & 116 & 6 & 54 \\
H & 72 & h & 104 & U & 85 & u & 117 & 7 & 55 \\
I & 73 & i & 105 & V & 86 & v & 118 & 8 & 56 \\
J & 74 & j & 106 & W & 87 & w & 119 & 9 & 57 \\
K & 75 & k & 107 & X & 88 & x & 120 &   &    \\
L & 76 & l & 108 & Y & 89 & y & 121 &   &    \\
M & 77 & m & 109 & Z & 90 & z & 123 &   &    \\
\hline
\end{tabular}
\end{center}

\end{multicols}
\end{document}
