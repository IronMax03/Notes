\documentclass[5pt]{article}
\usepackage{multicol,multirow}
\usepackage{graphicx} % Required for inserting images
\usepackage[margin=0.45cm]{geometry}
\usepackage{xcolor}
\definecolor{LightGray}{gray}{0.9}

\usepackage{minted}

\begin{document}
\begin{center}
     \Large{\textbf{C\texttt{++} Cheat Sheet}}\\
     \small{Class: CSCI 1300}\hfill\small{\textcopyright Maximilien Notz \the\year{}}
     \noindent\rule{20cm}{0.4pt}
\end{center}
\begin{multicols}{2}
\setcounter{secnumdepth}{0}


\section{General Reminders}
\begin{tabular}{ll}
\emph{code} & \emph{description}\\
\# include "myfile.h" & Include file.\\
\# include$<$cassert$>$ & Include asssert library.\\
assert(boolean); & Throw an error if the boolean is true.\\
rand() & Random int (\# include$<$cstdlib$>$).\\
abs($x$) & Returns $|x|$ (\# include$<$cstdlib$>$).\\
int(var) & Convert a primitive data type var to int.\\
$<<$setprecision(3)$<<$& Set decimal points (\#include$<$iomanip$>$)\\
int* myInt; & * means myInt work form a pointer.\\
\&var & Get mem addresse and pass by var by ref.\\
void myF() const & read only fonction\\
\end{tabular}


\section{Strings}
\begin{tabular}{ll}
\emph{code} & \emph{description}\\
str$[$i$]$ & Get or set the char at the index i.\\
str.length() & Return the number of characters.\\
str.substr(a,b) & Returns the substring from a to b.\\
str.find(subStr) & Retrun the start index of the substring\\
str.replace(i,l,str)& Replace substring from i with str\\
stoi(str) & Convert a string to int(\# include$<$string$>$).
\end{tabular}

\section{Arrays}
\begin{tabular}{c|c|c|c }
 0 & 1 & … & n \\ 
 \hline
 "Max" & "Tom" & … & arr$[$n$]$
\end{tabular}
This table illustrate the structure of an array of strings. Considering that n is equal to the number of element minus one. Arrays are a static data type.\\[6pt]
\begin{tabular}{ll}
\emph{code} & \emph{description}\\
int arr[4]; & Create a array of int and with 4 element.\\
int arr[4]=\{6,3\}; & \\
arr$[$i$]$ & Get or set the element at the index i.\\
\end{tabular}


\section{Vectors}
\begin{tabular}{ll}
\emph{code} & \emph{description}\\
\# include$<$vector$>$ & Include vector library.\\
vector$<$type$>$ V; & Instantiate a vector.\\
vector$<$type$>$ V(size); & Instantiate a vector from Array obj.\\
vector$<$type$>$ V\{6,3,3\}; & Instantiate a vector from Array.\\
V = vector$<$type$>$(); & Re-instantiate V\\
V.at(Index) & Returns the element at index i.\\
V.size() & Return the number of elements.\\
V.push\_back(Value)  & Add the new element at the end.\\
V.pop\_back() & Remove the last element.\\
V.clear() & Empty the vector.\\
V.insert(Index, Value) & Insert element at i.
\end{tabular}


\section{Structures}
\begin{minted}[bgcolor=LightGray]{C++}
struct myStruct {
    string param1;    // atribute 1
    double param2;   // atribute 2
}s1, s2;            // myStruct instances
\end{minted}
\begin{tabular}{ll}
\emph{code} & \emph{description}\\
myStruct Obj; & instantiate structure object.\\
Obj.param1 & Access param1 of Obj.\\
\end{tabular}


\section{Streams}
\begin{tabular}{ll}
\emph{code} & \emph{description}\\
\# include$<$fstream$>$ & Include stream library.\\
\# include$<$sstream$>$ & Include string stream library.\\
ifstream fin; & Instantiate a input stream.\\
ofstream fout; & Instantiate a output stream.\\
stringstream s(myStr); & Instantiate a string stream.\\
myStream.open("file.txt") & Open txt file whith the stream.\\
myStream.close() & Close the stream file.\\
getline(fin, line) & Get the next line from fin.\\
fout$<<$"hello"$<<$"World" & Output in stream "helloWorld".\\
fin$>>$var & Input from stream to var.\\
\end{tabular}
\subsection{clear buffer}
The buffer must be cleared after after getting an input from a stream if you input and output in the same file at the same time. 
\begin{minted}[bgcolor=LightGray]
{c++}
if(cin.fail() == true) {
    cout << "cin failed state";
    cin.clear(); 
    cin.ignore(1000, '\n'); 
}
\end{minted}

\section{Object Oriented Programing(OOP)}
\begin{minted}[bgcolor=LightGray]
{c++}
class myClasses {
    private:
        int param1;
    public:
        int param2;
        myClasses(int p1, int p2){ // constructor
            param1 = p1;
            param2 = p2;
        }

        myClasses(){ // default constructor
            param1 = -1;
        }

        string getParam1() { //getter
            return param1;
        }

        void setParam1(int p1) { // setter
            param1 = p;
        }
};
\end{minted}
\begin{tabular}{ll}
\emph{code} & \emph{description}\\
myClasses myObj(3,5); & Instantiate an myClasses type obj.\\
myClasses myObj; & Call the default constructor.\\
\end{tabular}
\newpage
\subsection{OOP With header file}
If you use a header the file wich contain the main function must include the header file.
\subsubsection{Header file(myHeader.h)}

\begin{minted}[bgcolor=LightGray]
{c++}
#ifndef MYCLASS_H //if no def for MyClass
#define MYCLASS_H //else

using namespace std;

class MyClass{
    public:
        MyClass(); //default constructor 
        MyClass(p1, p2); //parameterized constructor
        int publicAtribute;
        void myFunction() const;
    private:
        int privAtribute;
};
#endif
\end{minted}
\subsubsection{Class file(.cpp)}
\begin{minted}[bgcolor=LightGray]
{c++}
#include<iostream>
#include "myHeader.h"

MyClass::MyClass(){
    publicAtribute = 0;
    privAtribute = 0;
}

MyClass::MyClass(int p1, int p2){
    publicAtribute = p1;
    privAtribute = p2;
}

MyClass::void myFunction() const{
    // my code
}
\end{minted}

\section{Switch case}
\begin{minted}[escapeinside=§§, mathescape=true, bgcolor=LightGray]
{c++}
int x;
switch (x){
    case 0:
        /*Code in case 0*/
    break;
    §$\vdots$§
    case n:
        /*Code in case n*/
    break;
    default:
        /*Code if no case match*/
}
\end{minted}
\section{Important ASCII Conversions}
\begin{center}
\begin{tabular}{|ll|ll|ll|ll|ll|}
\hline
\tiny{\emph{ASCII}} &  \emph{int} & \tiny{\emph{ASCII}} &  \emph{int} &  \tiny{\emph{ASCII}} &  \emph{int} &  \tiny{\emph{ASCII}} &  \emph{int}&  \tiny{\emph{ASCII}} &  \emph{int}\\
 \hline
A & 65 & a & 97 & N & 78 & n & 110 & 0 & 48\\
B & 66 & b & 98 & O & 79 & o & 111 & 1 & 49\\
C & 67 & c & 99 & P & 80 & p & 112 & 2 & 50\\
D & 68 & d & 100 & Q & 81 & q & 113 & 3 & 51\\
E & 69 & e & 101 & R & 82 & n & 114 & 4 & 52\\
F & 70 & f & 102 & S & 83 & s & 115 & 5 & 53\\
G & 71 & g & 103 & T & 84 & t & 116 & 6 & 54\\
H & 72 & h & 104 & U & 85 & u & 117 & 7 & 55\\
I & 73 & i & 105 & V & 86 & v & 118 & 8 & 56\\
J & 74 & j & 106 & W & 87 & w & 119 & 9 & 57\\
K & 75 & k & 107 & X & 88 & x & 120 & &\\
L & 76 & l & 108 & Y & 89 & y & 121 & &\\
M & 77 & m & 109 & Z & 90 & z & 123 & &\\
\hline
\end{tabular}
\end{center}

\end{multicols}
\end{document}